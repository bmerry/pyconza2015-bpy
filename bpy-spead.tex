\documentclass{beamer}
\mode<presentation>
{
  \usetheme{Berkeley}
  \setbeamertemplate{footline}[frame number]
  \setbeamercovered{transparent}
}
\usepackage[english]{babel}
\usepackage[latin1]{inputenc}
\usepackage{times}
\usepackage[T1]{fontenc}
\usepackage{tikz}
\usepackage{booktabs}
\usepackage{listings}
\definecolor{lstpurple}{rgb}{0.5,0,0.35}
\definecolor{lstred}{rgb}{0.6,0,0}
\definecolor{lstgreen}{rgb}{0.25,0.5,0.35}
\definecolor{lstblue}{rgb}{0.25,0.35,0.75}
\lstset{language = Python,
    showstringspaces = false,
    columns = flexible,
    basicstyle = \ttfamily\footnotesize,
    keywordstyle = \color{lstpurple},
    keywordstyle = [2]\color{lstblue},
    commentstyle = \color{lstgreen},
    stringstyle = \color{lstred},
    numbers=left,numberstyle=\tiny,numbersep=1ex
}

\title[Boost.Python]{How I learn to stop worrying and love Boost.Python}

\subtitle{An adventure in high-performance networking}

\author{Bruce Merry}
% - Give the names in the same order as the appear in the paper.
% - Use the \inst{?} command only if the authors have different
%   affiliation.

\institute[SKA SA]{SKA South Africa}

\date{PyCon ZA 2015}

% If you have a file called "university-logo-filename.xxx", where xxx
% is a graphic format that can be processed by latex or pdflatex,
% resp., then you can add a logo as follows:

% \pgfdeclareimage[height=0.5cm]{university-logo}{university-logo-filename}
% \logo{\pgfuseimage{university-logo}}



% Delete this, if you do not want the table of contents to pop up at
% the beginning of each subsection:
\AtBeginSubsection[]
{
  \begin{frame}<beamer>{Outline}
    \tableofcontents[currentsection,currentsubsection]
  \end{frame}
}


% If you wish to uncover everything in a step-wise fashion, uncomment
% the following command: 

%\beamerdefaultoverlayspecification{<+->}


\begin{document}

\begin{frame}
  \titlepage
\end{frame}

\begin{frame}{Outline}
  \tableofcontents
  % You might wish to add the option [pausesections]
\end{frame}

\section{Motivation}

% - The problem
%   - MeerKAT
%   - SPEAD
%   - High performance networking, but different
% - The approach
%   - Why Python
%     - we're a Python shop; protocol is based on numpy
%   - Why C++ (and how hard is it really?)
% - Possible solutions
%   - Pure Python
%   - ctypes, cffi
%   - Cython
%     - Show the compiler error
%   - Python C API
%     - Yuck!
%   - Boost.Python
%     - Show a noddy example
%     - Downside: a runtime system library dependency
% - The interesting stuff
%   - GIL
%     - GIL background
%   - Lifetime management (custodian and ward)
%   - KeyboardInterrupt
%   - asyncio
%   - logging
%   - exceptions
%   - bytestring conversion

\section*{Summary}

\begin{frame}{Summary}
  \begin{itemize}
    \item Great for integrating C++ classes
    \item Can drop down to Python C API when needed
    \item Introduces a system library dependency
  \end{itemize}
\end{frame}

\end{document}
